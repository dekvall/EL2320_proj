\documentclass[a4paper,11pt]{article}
\usepackage[utf8]{inputenc}
\usepackage[T1]{fontenc}
\usepackage[english]{babel}
\usepackage{amsmath,amssymb,amsthm,amsfonts}
\usepackage{cleveref}
\usepackage{gensymb}
\usepackage{graphicx}
\usepackage[backend=biber,sorting=none]{biblatex}
\addbibresource{bib/bibliography.bib}
\graphicspath{{img/}}

\title{
	Multiple tennis ball tracking\\
	EL2320 Applied Estimation
}
\author{
	David Ekvall\\
	dekvall@kth.se\\
	\and
	Daniel Ahlberg\\
	ahlberg2@kth.se\\
}

\newcommand{\image}[3][width=0.9\columnwidth]{
	\begin{figure}[h!]
		\centering
	    \includegraphics[#1]{#2}
		\caption{#3}
		\label{fig:#2}
	\end{figure}
}

\begin{document}
\maketitle
\abstract{This is the abstract}

\section{Introduction}
1-2 pages
What is the problem you are trying to solve?
Tracking multiple bouncing tennis balls with occlusion. The tracking will be done using multiple particle filters and JPDA\cite{jaward2006multiple}.

\section{Related work}
1-2 pages \cite{reed2016generative}

\section{My method}
1-4 pages
\subsection{Overview}
\begin{enumerate}
	\item Simulate 1 ball
	\item Real world 1 ball
	\item Simulate multiple balls
	\item Real world multiple balls
\end{enumerate}


Tracking movement of the edges with particle filters. Hard to know where to cluster. 

\section{Implementation}
0-2 pages

\section{Results}
1-4 pages

\section{Conclusion}
0.5 - 1 page

\printbibliography

\end{document}
