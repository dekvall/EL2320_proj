\documentclass[a4paper,11pt]{article}
\usepackage[utf8]{inputenc}
\usepackage[T1]{fontenc}
\usepackage[english]{babel}
\usepackage{amsmath,amssymb,amsthm,amsfonts}
\usepackage{cleveref}
\usepackage{gensymb}
\usepackage{graphicx}
\usepackage[backend=biber,sorting=none]{biblatex}
\addbibresource{bib/bibliography.bib}
\graphicspath{{img/}}

\title{
	Create a map with hough transform\\
	EL2320 Applied Estimation
}
\author{
	David Ekvall\\
	dekvall@kth.se\\
	\and
	Daniel Ahlberg\\
	ahlberg2@kth.se\\
}

\newcommand{\image}[3][width=0.9\columnwidth]{
	\begin{figure}[h!]
		\centering
	    \includegraphics[#1]{#2}
		\caption{#3}
		\label{fig:#2}
	\end{figure}
}

\begin{document}
\maketitle
\abstract{This is the abstract}

\section{Introduction}
1-2 pages
What is the problem you are trying to solve?
Create a 2 dimensional map from a monocular camera feed by utilizing a particle filter on a hough transform. It's a simple occupancy grid problem because we know where we are. The problem is to acquire the position of the floor and walls to detect if there is an obstacle there. 

\section{Related work}
1-2 pages \cite{reed2016generative}

\section{My method}
1-4 pages
Vertical edges are important landmarks, since longer wall segments will indicate forward movement. A longer segment is actually more concentrated. 

Tracking movement of the edges with particle filters. Hard to know where to cluster. 

\section{Implementation}
0-2 pages

\section{Results}
1-4 pages

\section{Conclusion}
0.5 - 1 page

\printbibliography

\end{document}
